\chapter{Numerics}
According to \cite{erlebacher1985finite} the Gaus divergenze theorem can be rewritten as
\begin{equation}
  \label{eqn:gauss}
  \int_V \nabla \vec{F} \,\, \mathrm{d}^n V = \oint_S \vec{F} \cdot \vec{n} \,\, \mathrm{d}^{n-1}S
\end{equation}
Using $\nabla f$ for $\vec{f}$ Eqn.~(\ref{eqn:gauss}) can be rewritten as
\begin{equation}
  \int_V \nabla^2 f \,\, \mathrm{d}^n V = \oint_S (\nabla f) \cdot \vec{n} \,\, \mathrm{d}^{n-1}S
\end{equation}
For the numerical solution a cell wise desrcitization is made.
Assuming that $\nabla^2 f$ is constants over a cell the equation above reduces to
\begin{equation}
  V_C \cdot \nabla^2 f = \sum_i^N (\nabla f) \cdot \vec{n}_i \cdot S_{i}.
\end{equation}

\section{Gradient}
The base equation is given as
\begin{equation}
    \int_A \nabla f \, \mathrm{d}A = \oint_S f \cdot \vec{n} \, \mathrm{d}S
\end{equation}
The left hand side of the equation can be rewritten as
\begin{equation}
    \int_A \nabla f \, \mathrm{d}A = A_c \cdot \nabla f_c
\end{equation}
The right hand side can be written as
\begin{equation}
    \oint_S f \cdot \vec{n} \, \mathrm{d}S = \sum_{i = 1}^N \frac{f_{i + 1} + f_i}{2} \cdot \vec{n}_i \cdot |\vec{r}_{i+1, i}|
\end{equation}
where $\vec{r}_{i+1, i} = \vec{x}_{i + 1} - \vec{x}_i$ and it can be written
\begin{equation}
    \vec{n}_i \cdot |\vec{r}_{i+1, i}| = M_{rot} \cdot \vec{r}_{i+1, i}
\end{equation}
where
\begin{equation}
    M_{rot} = 
    \begin{bmatrix}
       \cos(\theta) & -\sin(\theta) & 0 \\
       \sin(\theta) &  \cos(\theta) & 0 \\
       0 & 0 & 1
     \end{bmatrix}
\end{equation}
with $\theta = -\pi /2$
\begin{equation}
    M_{rot} = 
    \begin{bmatrix}
       0 & 1 & 0 \\
       -1 &  0 & 0 \\
       0 & 0 & 1
     \end{bmatrix}
\end{equation}
Since the integral mus be performed over a closed loop
\begin{equation}
    \begin{split}
         \nabla f_c &= \frac{1}{2 \cdot A_c} \cdot \sum_{i = 1}^N (f_{i + 1} + f_i) \cdot M \cdot \vec{r}_{i + 1,i} \\
         &= \frac{1}{2 \cdot A_c} \cdot \left[ (f_{i + 1} + f_i) \cdot M \cdot (\vec{x}_{i + 1} - \vec{x}_i) + \dots + (f_{i} + f_{i - 1}) \cdot M \cdot \vec{x}_{i} - \vec{x}_{i - 1} \right] \\
         &= \frac{1}{2 \cdot A_c} \cdot \left[ f_i \cdot M \cdot (\vec{x}_{i + 1} - \vec{x}_{i - 1}) + f_{i + 1} \cdot M \cdot (\vec{x}_{i + 2} - \vec{x}_{i}) + \dots \right]
    \end{split}
\end{equation}
Thus
\begin{equation}
    \label{eqn:gradient}
    \nabla f_c = \frac{1}{2 \cdot A_c} \cdot \sum_{i = 1}^N f_i \cdot M \cdot \vec{r}_{i + 1,i - 1}
\end{equation}

\section{Normal derivative}
For the normal derivative at the cell wall $w$ with the normal vector $n_w$ one can write
\begin{equation}
    \underbrace{\nabla f_c \cdot \vec{n}_w}_{\nabla_{w} f_c} = \frac{1}{2 \cdot A_c} \cdot \sum_{i = 1}^N f_i \cdot (M \cdot \vec{r}_{i + 1,i - 1}) \cdot \vec{n}_w
\end{equation}
To find the derivative on a node surrounded by $K$ neighbouring cells with the total surface $A_{tot} = \sum_{c=1}^K A_c$
\begin{equation}
    \sum_{c=1}^K \frac{A_c}{A_{tot}} \cdot \nabla_{c, w} f_c = \sum_{c=1}^K \frac{A_c}{A_{tot}} \cdot \frac{1}{2 \cdot A_c} \cdot \sum_{i = 1}^N f_{c, i} \cdot (M \cdot \vec{r}_{c, i + 1,i - 1}) \cdot \vec{n}_{c,w}
\end{equation}
using $\overline{A_c} = \frac{A_c}{A_{tot}}$
\begin{equation}
    \sum_{c=1}^K \overline{A_c} \cdot \nabla_{c, w} f_c = \frac{1}{2 \cdot A_{tot}} \cdot\sum_{c=1}^K  \sum_{i = 1}^N f_{c, i} \cdot (M \cdot \vec{r}_{c, i + 1,i - 1}) \cdot \vec{n}_{c, w}
\end{equation}

\section{Laplace}
The integral is made over a pseudo cell which connects the barycentres of adjacent cells
\begin{equation}
    \begin{split}
        \nabla^2 f \cdot \underbrace{\sum_{b=1}^N A_b}_{A_{b, tot}}  &=  \sum_{b = 1}^N \nabla f \cdot M \cdot \vec{r}_{b + 1,b} \\
        &= \sum_{b = 1}^N \frac{A_b \cdot\nabla f_{b} + A_{b+ 1} \cdot \nabla f_{b + 1}}{A_{b+1, b}} \cdot M \cdot \vec{r}_{b + 1,b} 
    \end{split}
\end{equation}
Using Eqn.~(\ref{eqn:gradient}) 
\begin{equation}
    \nabla^2 f \cdot A_{b, tot}  = \sum_{b = 1}^N \frac{1}{2 \cdot A_{b+ 1, b}} \cdot \left( \sum_{q=0}^1\sum_{i = 1}^3 f^{b + q}_{i} \cdot M \cdot \vec{r}^{\:b + q}_{i + 1,i - 1}  \right) \cdot M \cdot \vec{r}_{b + 1,b}
\end{equation}
As the gradient is considered constant inside the cell
\begin{equation}
    \nabla^2 f \approx \frac{1}{2 \cdot A_{b, tot}} \cdot\sum_{b = 1}^N \frac{1}{A_{b+ 1, b}} \cdot \left( \sum_{q=0}^1\sum_{i = 1}^3 f^{b + q}_{i} \cdot M \cdot \vec{r}^{\:b + q}_{i + 1,i - 1}  \right) \cdot M \cdot \vec{r}_{b + 1,b}
\end{equation}
