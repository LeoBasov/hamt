\chapter{Heat Transfer}
The base equation for the heat transfer is the balance equation of the form
%----------------------------------------------------------------
\begin{equation}
\label{eqn:balance}
    \pdv{\rho}{t} + \nabla \vec{j} = f(\vec{r}, t)
\end{equation}
%----------------------------------------------------------------
where $\rho$ is the density of the quantity in question, $\vec{j}$ is the flux, $\vec{r}$ is the position vector and $t$ is time.
For heat transfer problem we define $\rho$ as the volumetric heat density $q$ which is defined as
%----------------------------------------------------------------
\begin{equation}
\label{eqn:heat_density}
    q = \frac{Q}{V} = c \, \rho \, T
\end{equation}
%----------------------------------------------------------------
$\vec{j}$ is equvalent to the heat flux $\vec{\dot{q}}$
%----------------------------------------------------------------
\begin{equation}
\label{eqn:heat_flux}
    \vec{\dot{q}} = -\lambda \, \nabla T
\end{equation}
%----------------------------------------------------------------
and the source$f(\vec{r}, t)$ is the volumetrix heat source $\dot{q}_v(\vec{r}, t)$.
Inserting \ref{eqn:heat_density} and \ref{eqn:heat_flux} in \ref{eqn:balance} leads to the heat transfer equation
%----------------------------------------------------------------
\begin{equation}
\label{eqn:heat_transfer}
    \pdv{(c \, \rho \, T)}{t} + \nabla (-\lambda \, \nabla T ) = \dot{q}_v(\vec{r}, t)
\end{equation}
%----------------------------------------------------------------
where the quantities are defined as seen in table \ref{tab:parameters}.
%----------------------------------------------------------------
\begin{table}[ht]
\centering
\caption{Heat Transfer - parameters}
\label{tab:parameters}
\begin{tabular}{lll}
\hline
\hline
Quantitiy & Description & Unit\\
\hline
$c$ & Specific heat capacity & $\si{\joule\per\kilogram\per\kelvin}$\\
$\rho$ & Density & $\si{\kilogram\per\meter\cubed}$\\
$T$ & Temperature & $\si{\kelvin}$\\
$\lambda$ & Thermal conductivity & $\si{\watt\per\meter\per\kelvin}$\\
$\dot{q}_v$ & Volumetric heat source & $\si{\watt\per\meter\cubed}$\\
\hline
\hline
\end{tabular}
\end{table}
%----------------------------------------------------------------

\section{Homogeneous Heat Transfer Equation}
Aussuming no time dependance $\pdv{(c \, \rho \, T)}{t} = 0$ and no heat source $\dot{q}_v(\vec{r}, t) = 0$ Eqn.~(\ref{eqn:heat_transfer}) reduces to
%----------------------------------------------------------------
\begin{equation}
\label{eqn:homogeneous_heat_transfer}
    \nabla (\lambda \, \nabla T )  = 0
\end{equation}
%----------------------------------------------------------------
where $\lambda = \mathrm{fn}(\vec{r})$.
According to the prodcut rules for vector calculus Eqn.~(\ref{eqn:homogeneous_heat_transfer}) evaluates to
\begin{equation}
	\nabla (\lambda \, \nabla T ) = \nabla T \cdot (\nabla \lambda) +\lambda \left(\nabla \cdot \nabla T \right)
\end{equation}
\begin{equation}
	\nabla (\lambda \, \nabla T ) = \nabla \lambda \cdot \nabla T + \lambda \cdot \nabla^2 T
\end{equation}
which leads to
\begin{equation}
	\nabla \lambda \cdot \nabla T + \lambda \cdot \nabla^2 T = 0.
\end{equation}

\subsection{2D Cartesian Formulation}
Evaluating Eq.~(\ref{eqn:homogeneous_heat_transfer}) for two variables in cartesian form leads to
%----------------------------------------------------------------
\begin{align}
    \nabla (\lambda \, \nabla T )  &= \pdv{}{x} \left(\lambda \pdv{T}{x} \right) + \pdv{}{y} \left(\lambda \pdv{T}{y} \right) \\
    &= \pdv{\lambda}{x} \pdv{T}{x} + \lambda \pdv[2]{T}{x} + \pdv{\lambda}{y} \pdv{T}{y} + \lambda \pdv[2]{T}{y} \\
    &= \pdv{\lambda}{x} \pdv{T}{x} + \pdv{\lambda}{y} \pdv{T}{y} + \lambda \pdv[2]{T}{x} + \lambda \pdv[2]{T}{y} \\
    &= \nabla \lambda \cdot \nabla T + \lambda \cdot \nabla^2 T
\end{align}
%----------------------------------------------------------------

\subsection{2D Cylindrical Formulation}
Evaluating Eq.~(\ref{eqn:homogeneous_heat_transfer}) for two variables in cartesian form leads to
%----------------------------------------------------------------
\begin{align}
    \nabla (\lambda \, \nabla T )  &= \pdv{}{r} \left(\lambda \pdv{T}{r} \right) + \frac{1}{r}\pdv{}{\phi} \left(\lambda \frac{1}{r} \pdv{T}{\phi} \right)  \\
    &= \pdv{\lambda}{r} \pdv{T}{r} + \lambda \pdv[2]{T}{r} + \frac{1}{	r} \left[ \pdv{\lambda}{\phi}\frac{1}{r} \pdv{T}{\phi} + \lambda\frac{1}{r} \pdv[2]{T}{\phi}  \right] \\
    &= \pdv{\lambda}{r} \pdv{T}{r} + \frac{1}{r} \pdv{\lambda}{\phi}\frac{1}{r} \pdv{T}{\phi} + \lambda \left[\pdv[2]{T}{r} + \frac{1}{r^2}\pdv[2]{T}{\phi} \right] \\
    &= \nabla \lambda \cdot \nabla T + \lambda \left[\pdv[2]{T}{r} + \frac{1}{r^2}\pdv[2]{T}{\phi} \right]
\end{align}
%----------------------------------------------------------------

\section{Finite-Difference Scheme - Triangular Unordered 2D Grid}
Using a second order Taylor serias expansion around a arbitray point in space for two variables on can write
%----------------------------------------------------------------
\begin{equation}
  \mathrm{T} \, f(x ; a) = f(a) + \sum_{j=1}^2 \pdv{f(a)}{x_j} (x_j - a_j) + \frac{1}{2} \sum_{j=1}^2 \sum_{k=1}^2 \pdv{f(a)}{x_j}{x_k} (x_j - a_j) (x_k - a_k) + O(\Delta^3)
\end{equation}
%----------------------------------------------------------------


\section{Finite-Difference Scheme - Homogeneous Equidistant 2D Grid}
\subsection{Boundary Conditions}
For all boundary condition is present the dirichlet boundary condtions always orride other boudnary conditions on contact nodes.
If two cells with different dirichlet boundary condtions touch, a arichmetical mean is taken.

\subsubsection{Corner Boundary Conditions}
Examples buttom left corner
%----------------------------------------------------------------
\begin{equation}
  T_{i+1, j} - T_{i, j} = \Delta x_1 \, c_1
\end{equation}
%----------------------------------------------------------------
\begin{equation}
  T_{i, j + 1} - T_{i, j} = \Delta x_2 \, c_2
\end{equation}
%----------------------------------------------------------------
By adding the two functions one gets
%----------------------------------------------------------------
\begin{equation}
  T_{i+1, j} + T_{i, j + 1} - 2 T_{i, j} = \Delta x_1 \, c_1 + \Delta x_2 \, c_2.
\end{equation}
%----------------------------------------------------------------

\subsubsection{Radiation Boundary Conditions}
%----------------------------------------------------------------
\begin{equation}
    \vec{\dot{q}}_r = \epsilon \, \sigma \, T^4 \, \vec n
\end{equation}
%----------------------------------------------------------------
Using the Taylor series
%----------------------------------------------------------------
\begin{equation}
    \mathrm{T} \, f(x;a) = \sum^{\infty}_{n = 0} \frac{f^{(n)}(a)}{n!}(x - a)^n
\end{equation}
%----------------------------------------------------------------
we get the linearized equation for the heat flux due to radiation
%----------------------------------------------------------------
\begin{equation}
    \mathrm{T} \, \dot{q}_r(T;T_0) = (4 \, \epsilon \, \sigma \, T_0^3) T - 3 \, \epsilon \, \sigma \, T_0^4.
\end{equation}
%----------------------------------------------------------------
Using eqation (\ref{eqn:heat_flux})
%----------------------------------------------------------------
\begin{equation}
    -\lambda \pdv{T}{x} = \mathrm{T} \, \dot{q}_r(T;T_0)
\end{equation}
%----------------------------------------------------------------
leads to the discitized, linear equation
%----------------------------------------------------------------
\begin{equation}
    \left( 1 + 4\,k\,T_0^3 \right) T_i - T_{i - 1}= 3\,k\,T_0^4
\end{equation}
%----------------------------------------------------------------
where
%----------------------------------------------------------------
\begin{equation}
    k = \frac{\epsilon\,\sigma\,\Delta x}{\lambda}
\end{equation}
%----------------------------------------------------------------
\begin{table}[ht]
\centering
\caption{Radiation Boundary Condition - parameters}
\begin{tabular}{lll}
\hline
\hline
Quantitiy & Description & Unit\\
\hline
$\epsilon$ & Emissivity factor & $-$\\
$\sigma$ & Stefan-Boltzmann constant & $\si{\watt\per\meter\squared\per\kelvin\tothe{4}}$\\
$\Delta x$ & Spatial step & $\si{\meter}$\\
\hline
\hline
\end{tabular}
\end{table}
%----------------------------------------------------------------

\subsection{Cartesian Coordinates}
The homogeneous heat equation in cartesian coordinates is expressed as
%----------------------------------------------------------------
\begin{equation}
\label{eqn:homogeneous_heat_transfer_cartesian}
    \lambda\pdv[2]{T}{x} + \lambda\pdv[2]{T}{y} + \lambda\pdv[2]{T}{z} = 0
\end{equation}
%----------------------------------------------------------------
By assuming only the $x$ and $y$ directions equation (\ref{eqn:homogeneous_heat_transfer_cartesian}) reduces to
%----------------------------------------------------------------
\begin{equation}
  \label{eqn:fdm_equi_cart_2d}
  \lambda\pdv[2]{T}{x} + \lambda\pdv[2]{T}{y} = 0.
\end{equation}
%----------------------------------------------------------------
Using equation (\ref{eqn:fdm_2}) equation (\ref{eqn:fdm_equi_cart_2d}) can be discritizised as
%----------------------------------------------------------------
\begin{equation}
  \lambda_{i+1, j}T_{i + 1, j}  + \lambda_{i, j + 1}T_{i, j + 1} - \\
  \lambda_{tot} T_{i, j} + \\
  \lambda_{i-1, j}T_{i-1, j}  + \lambda_{i, j -1}T_{i, j - 1} = 0
\end{equation}
%----------------------------------------------------------------
where
%----------------------------------------------------------------
\begin{equation}
  \lambda_{tot} = \lambda_{i+1, j} + \lambda_{i-1, j} + \lambda_{i, j+1} + \lambda_{i, j-1}.
\end{equation}
%----------------------------------------------------------------
In case any of the $\lambda$ paramters is to be taken on a boundary between two segments, a mean between to is to take.
The correctness of this assumptions can be proven by setting up 4 equations around a center node using Fourrier's law.
For $\lambda = const$ the equation reduces to
%----------------------------------------------------------------
\begin{equation}
  T_{i + 1, j}  + T_{i, j + 1} - 4 \, T_{i, j} + T_{i-1, j}  + T_{i, j - 1} = 0
\end{equation}
%----------------------------------------------------------------

\subsection{Cylinder Coordinates}
while in cylinder coordinates as
%----------------------------------------------------------------
\begin{equation}
\label{eqn:homogeneous_heat_transfer_cylinder}
    \lambda\frac{1}{r}\pdv{}{r}\left(r \pdv{T}{r}\right) + \lambda\frac{1}{r^2}\pdv[2]{T}{\phi} + \lambda\pdv[2]{T}{z} = 0.
\end{equation}
%----------------------------------------------------------------

\subsection{Verification}
Heat codnuctivity trhough a layerd wall
%----------------------------------------------------------------
\begin{equation}
  \dot{q} = \left( \sum_{i = 1}^{N} \frac{\Delta x_i}{\lambda_i} \right)^{-1}\Delta T
\end{equation}
%----------------------------------------------------------------
